\documentclass{report}

\usepackage{markdown}

\begin{document}

\title{SSQA : Jalon 1}

\author{Louis-Vincent CAPELLI \and Alexandre THEISSE \and Tom SARTORI \and Raphaël TURCOTTE}
\date{\today}
\maketitle

\chapter{Organisation}
\section{Réunions}
Nous avons décidé de nous réunir tous les mardis à 15h20 
afin de profiter du fait d'être tous présents à l'université. 
Nous avons également parfois ajouté des créneaux de réunions,
par exemple le samedi, pour faire un rapide point sur l'avancement
du projet en visioconférence sur Discord.

\section{Création des lots de travail}
Lors de ces réunions, nous abordons les différents points prévus,
nous parcourons les différentes tâches qui étaient à réaliser
et surtout nous définissons ensemble les lots de travail
pour la semaine à venir, la difficulté étant de répartir correctement
la charge de travail entre les membres du groupe et de permettre à
tous le monde de travailler en simultané sur le projet.

\chapter{Répartition des tâches}
\section{Travail simultané}
Afin que chacun puisse progresser et aussi parce que nous y étions
parfois contraints, de nombreuses tâches ont été assignées à
l'ensemble du groupe. Par exemple, la modification du diagramme
issu de l'EPP\_V1 en vue du passage à la V2 a été réalisée par
l'ensemble du groupe individuellement. Nous avons pu ensuite 
comparer nos résultats et en discuter lors des réunions afin
de tirer le meilleur du travail de chacun.

\section{Lots individuels}
Nous avons essayé, lorsque les tâches s'y prétaient, de les
répartir entre les membres du groupe. Par exemple, la création
des scripts SQL et la rédaction des tests ou encore la rédaction
des différents documents ont ou être partagés afin de gagner
du temps.


\chapter{Problèmes rencontrés}
\section{Prise en main du projet}

Nous avons perdu beaucoup de temps au début du projet à essayer
de comprendre ce qui était concrètement attendu de nous.
Comme on peut le voir dans les Compte-Rendus de réunions,
nous avons compris assez tardivement que la partie EPP\_V1 du
projet était à considérer comme déjà réalisée et qu'il fallait
directement partir de ce qui avait été fait pour faire la
transition vers la V2.
\\\\
Certains membres ont également rencontré des difficultés à
se connecter à la base de données.

\section{Difficultés à travailler en simultané}
La partie analyse et synthèse des données de l'énoncé
afin de mettre au point un schéma de base de données ainsi que
la normalisation de ce dernier étant des processus assez
linéaires, il était difficile de travailler en simultané sur
ces tâches. Nous avons donc dû nous répartir le travail
en fonction de l'avancement de chacun ce qui a mener à des
pertes de temps et des blocages dans l'avancement du projet.

\section{Manque de temps}
Nous avons également eu du mal à faire concorder nos emplois
du temps et tous les membres du groupes n'étaient pas disponibles
pour rédiger les scripts SQL et rédiger le rapport 
au moment où nous avions fini de mettre au point le schéma de
base de données.
\\\\
Finalement, nous avons pu rédiger l'ensemble des scripts
et une majorité des test, mais la mise au propre des SCBD et SEM
ainsi que la rédaction du rapport ont dû être réalisées
dans l'urgence juste avant la date de rendu.

\chapter{Conclusion}
Ce premier jalon nous a permis de nous familiariser avec le
projet et de nous rendre compte de la charge de travail
nécessaire pour le mener à bien. Nous avons également pu
mettre en place une organisation de travail que nous pourrons
conserver pour le reste du projet.
\\\\
Pour les jalons suivant, nous allons pouvoir nous concentrer
sur une meilleure répartition des tâches, que ce soit entre
les membres du groupe et dans le temps, afin de ne pas avoir
à travailler dans l'urgence.

\chapter*{Annexes - Compte-rendus de réunion}

\markdownInput{./CR1.md}

\markdownInput{./CR2.md}

\markdownInput{./CR3.md}
\end{document}
