\documentclass{article}
\usepackage{graphicx}
\usepackage{textgreek}

\begin{document}
\title{Spécifiation de conception de la base de données : Jalon 2}

\author{Louis-Vincent CAPELLI \and Alexandre THEISSE \and Tom SARTORI \and Raphaël TURCOTTE}
\date{\today}
\maketitle
\newpage

\tableofcontents
\newpage

\section{Introduction}
\subsection*{Objet et portée du document}
Ce document a pour but de documenter la création d'une API sur la base de données
SSQA afin de répondre aux exigences formulées dans la
spécifiation des exigences du modèle pour le jalon 2.
Il s'adresse à toute personne qui pourrait avoir à travailler sur la base de données
SSQA à l'avenir.

\section{Présentation générale du résultat}
L'ensemble de l'API est en PostgreSQL comme la base de données SSQA.
\\\\
Nous avons décidé de créer les fonctions atomiques correspondant aux primitives 
fondamentales d'ÉMIR dans le schéma original de la base de données qui est privé.
Les fonctions de production de rapport, elles sont dans un nouveau schéma public
afin de permettre à l'utilisateur de les utiliser sans avoir à connaître le
schéma privé.


\section{Choix de conception}
\subsection{Primitives fondamentales}
\paragraph{Description} Implémenter les primitives fondamentales d'ÉMIR
pour les tables Territoire, Station, Variable et Mesure.

\paragraph{Solution choisie}

\paragraph{Solution alternative}

\subsection{Requête 1}
\paragraph{Description} Créer une fonction permet à l'utilisateur
d'effectuer la requête suivante : "Quel est l'IQA découlant des mesures
prises par la station du quartier universitaire de la ville de Sherbrooke,
le 2016-06-12 ?"

\paragraph{Solution choisie}

\paragraph{Solution alternative}

\subsection{Requête 2}
\paragraph{Description} Créer une fonction permet à l'utilisateur
d'effectuer la requête suivante : "Quelles sont les stations du territoire
de la ville de Sherbrooke ?"

\paragraph{Solution choisie}

\paragraph{Solution alternative}

\subsection{Requête 3}
\paragraph{Description} Créer une fonction permet à l'utilisateur
d'effectuer la requête suivante : "Quels sont les quartiers de la ville
de Sherbrooke qui dépassent la norme canadienne de la qualité de l'air
au moins n fois par année ?"

\paragraph{Solution choisie}

\paragraph{Solution alternative}

\subsection{Requête 4}
\paragraph{Description} Créer une fonction permet à l'utilisateur
d'effectuer la requête suivante : "Dans un territoire donné, au cours de
l'année 2021, quels sont les polluants qui ont dépassé la valeur de 
référence d'au moins 10\% ?"

\paragraph{Solution choisie}

\paragraph{Solution alternative}

\subsection{Requête 5}
\paragraph{Description} Créer une fonction permet à l'utilisateur
d'effectuer la requête suivante : "Quels sont les territoires ayant au
moins trois fois par mois un mauvais IQA au  cours des deux dernières
années ?"

\paragraph{Solution choisie}

\paragraph{Solution alternative}

\end{document}
