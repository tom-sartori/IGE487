\documentclass{report}

\usepackage{markdown}

\begin{document}

\title{SSQA : Jalon 2}

\author{Louis-Vincent CAPELLI \and Alexandre THEISSE \and Tom SARTORI \and Raphaël TURCOTTE}
\date{\today}
\maketitle

\chapter{Organisation}
\section{Réunions}
Nous avons fait une réunion directement après la réception du sujet
afin de nous mettre d'accord sur les grandes lignes du projet et
de s'assurer de notre compréhension de l'énoncé. Nous avons ainsi pu
nous mettre d'accord sur les outils à utiliser et sur la manière
dont nous allions nous organiser pour ce jalon. (cf. Annexe CR4)

Nous nous sommes ensuite réunis à nouveau quelques jours plus tard
afin de créer les lots de travail et de nous les répartir. (cf. Annexe CR5)


\chapter{Solutions aux problèmes du Jalon 1}
\section{Prise en main du projet}
Nous avons pris le temps d'éclaircir les points de l'énoncé qui
nous paraissaient obscurs avec M. Lavoie avant de commencer à
travailler sur le projet, cela concernait notamment les technologies
à utiliser ou non et la manière de rédiger les documents de
conception et de spécification du Jalon 2 en fonction de ce qui
avait été fait pour le Jalon 1.

\section{Travail simultané}
Nous nous sommes appliqués à créer des lots de travail qui
pouvaient être réalisés en parallèle afin de gagner du temps.
Par exemple, nous avons commencer la rédaction des fonctions de
production de rapport avant d'avoir fini la création des fonctions
atomiques, dans le but de les refactoriser par la suite si besoin.

Nous avons également fait une point ensemble sur les exigences
du projet afin de pouvoir commencer à travailler sur les différents
lots pendant la rédaction du SEM.

\section{Manque de temps}
Les solutions mises en place pour les problèmes précédents et notre expérience
du Jalon 1 nous
ont permis de gagner du temps et d'être efficace sur
la rédaction du rapport et la mise au propre des SCBD, SEM
et Rapport dans les derniers jours avant la date de rendu.

\chapter{Problèmes rencontrés}
\section{Mise en commun du travail effectué}
La création de lots de travail les plus indépendants possibles
nous a permis de travailler en parallèle sur le projet, mais
a aussi eu comme conséquence de rendre plus difficile la compréhension
par chacun du travail effectué par les autres. L'étape de refactorisation
des fonctions de production de rapport en a donc pâti puisque nous nous
sommes rendus compte très tard que les fonctions de production de rapport
qui avaient été rédigées étaient incorrectes.

Il aurait peut-être été bénéfique de faire une réunion de plus
pour que chacun puisse présenter son travail aux autres et qu'il puisse
être validé par le reste du groupe et ainsi répérer les erreurs plus tôt.

\chapter{Conclusion}
Nous sommes satisfaits du travail que nous avons effectué pour ce jalon.
Nous avons réussi à nous organiser efficacement pour travailler en parallèle
et à nous répartir les tâches de manière équilibrée, ce qui nous a permis
de gagner du temps et de ne pas avoir à travailler dans l'urgence
les derniers jours avant la date de rendu.

Nous avons mis en place des solutions pour les problèmes que nous avions
rencontrés lors du Jalon 1, ce qui nous a permis de ne pas les retrouver
pour ce jalon et nous tâcherons de continuer à les appliquer pour les
jalons suivants ainsi que d'autres solutions aux problèmes que nous
avons rencontrés pour ce jalon.

\chapter*{Annexes - Compte-rendus de réunion}

% \markdownInput{./CR4.md}

% \markdownInput{./CR5.md}
\end{document}