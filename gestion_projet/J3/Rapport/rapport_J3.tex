\documentclass{report}

\usepackage{markdown}
\usepackage{indentfirst}
\usepackage[margin=1.5in]{geometry}

\begin{document}

\title{SSQA : Rapport de jalon 3}

\author{Louis-Vincent CAPELLI \and Alexandre THEISSE \and Tom SARTORI \and Raphaël TURCOTTE}
\date{\today}
\maketitle

\chapter{Organisation}
\section{Réunions}
Nous avons fait une réunion directement après la réception du sujet afin de nous mettre
d'accord sur les grandes lignes du projet et de s'assurer de notre compréhension de l'énoncé.
Nous avons ainsi pu nous mettre d'accord sur la manière dont nous allions nous organiser
pour ce jalon et sur les solutions que nous allions mettre en place pour les problèmes pour
répondre aux exigences du client.

Nous avons décidé, pour les raisons évoquées dans la partie suivante, de ne pas scinder
le travail en lots et de travailler ensemble et simultanément sur chaque tâche.
Comme nous disposions de plus de temps pour ce jalon, nous avons décidé de maintenir le
créneau du mercredi après-midi qui nous servait à faire le point sur l'avancement du projet.
Ce créneau a été utilisé pour faire des réunions de travail au cours desquelles nous avons
pu réfléchir ensemble à la manière de répondre aux exigences et également profiter de la
présence du client (et/ou du professeur selon la casquette qu'il revêtait) pour lui poser des
questions sur le sujet et lui soumettre nos idées.


\chapter{Solutions aux problèmes des jalons précédents}

\section{Travail simultané (Jalon 1)}
Ne trouvant pas de solution satisfaisante pour ce problème, dans le cadre du jalon 3 qui
nous semblait ne pas contenir assez de tâches distinctes, nous avons décidé de ne pas scinder
le travail en lots de travail et de travailler ensemble sur chaque tâche.

\section{Mise en commun du travail effectué (Jalon 2)}
Nous nous étions rendu compte lors du jalon 2 que nous avions pris trop de temps à mettre
en commun le travail effectué par chacun et à le vérifier car nous étions trop concentrés sur
nos propres tâches. Le mode de fonctionnement que nous avons adopté pour ce jalon nous
a permis de ne pas rencontrer ce problème car tous les membres étaient présents lors des réunions de
travail et pouvaient avoir un regard critique sur le travail effectué par les autres.

\chapter{Conclusion}
Nous sommes satisfaits du travail que nous avons effectué pour ce jalon. Nous avons réussi
à nous organiser efficacement et en adéquation avec les spécificités de celui-ci, ce qui nous a
permis de le mener à bien dans les temps.

Nous avons mis en place des solutions aux problèmes que nous avions rencontrés lors
des jalons précédents afin de ne pas les rencontrer à nouveau et de nous
concentrer sur les exigences du client. Ce jalon qui marque la fin du projet a été l'occasion
de mettre en pratique les notions vues en cours et de nous familiariser avec les outils utilisés
dans le monde professionnel mais également de tirer des enseignements sur notre manière
de travailler en groupe et de nous organiser.

\end{document}